\documentclass[12pt,a4paper]{report}
\usepackage[utf8]{inputenc}
\usepackage{graphicx}
\usepackage{amsmath}
\usepackage{amssymb}
\usepackage{lipsum}
\usepackage{textcomp}
\usepackage{fixltx2e}
\usepackage{fancyhdr}
\def\SPSB#1#2{\rlap{\textsuperscript{{#1}}}\SB{#2}}
\def\SP#1{\textsuperscript{{#1}}}
\def\SB#1{\textsubscript{{#1}}}
\usepackage[svgnames]{xcolor}
\usepackage{caption}
\usepackage{subcaption}

\newcommand*{\rotrt}[1]{\rotatebox{90}{#1}} % Command to rotate right 90 degrees
\newcommand*{\rotlft}[1]{\rotatebox{-90}{#1}} % Command to rotate left 90 degrees

\newcommand*{\titleBC}{\begingroup % Create the command for including the title page in the document
\centering % Center all text

\def\CP{\textit{\Large Simulating argon }} % Title

\settowidth{\unitlength}{\CP} % Set the width of the curly brackets to the width of the title
{\color{CadetBlue}\resizebox*{\unitlength}{\baselineskip}{\rotrt{$\}$}}} \\[\baselineskip] % Print top curly bracket
\textcolor{Black}{\CP} \\[\baselineskip] % Print title
{\color{Grey}\large Delft University of Technology} \\ % Tagline or further description
{\color{CadetBlue}\resizebox*{\unitlength}{\baselineskip}{\rotlft{$\}$}}} % Print bottom curly bracket

\vfill % Whitespace between the title and the author name

% Author name
{\large\textbf{Ludwig Rasmijn}}\\
{\large\textbf{Sebastiaan Lokhorst }}\\
{\large\textbf{Shang-Jen Wang (4215974)}}\\
\bigskip
{\large\textbf{Supervisors:}}\\
{\large\textbf{ }}
\vfill % Whitespace between the author name and the publisher logo
\pagestyle{empty}
2015 % Year published

\endgroup}

\begin{document}
\pagestyle{empty}
\pagenumbering{gobble}
\titleBC
\newpage
\begin{abstract}
\noindent

\end{abstract}

\newpage
\tableofcontents

\newpage
\pagestyle{fancy}
% Clear the header and footer
\fancyhead{}
\fancyfoot{}
\renewcommand{\headrulewidth}{0pt}
% Set the right side of the footer to be the page number
\fancyfoot[C]{\thepage}
\pagenumbering{arabic}
\listoffigures

\chapter{Introduction}

\chapter{Theory}
Four questions have to be answered in order to simulate argon, these are:
\begin{enumerate}
 \item What is the interaction between the argon atoms?
 \item How do the argon atoms move around?
 \item What are the boundary conditions?
 \item What are the initial conditions?
\end{enumerate}
These will be answered in the following sections.
\section{Lennard-Jones potential}

\section{Velocity Verlet}
In this section the question ``How do the argon atoms move around?" will be answered.\\
Newton's equation of motion can be used to describe the motion of argon:
\begin{equation}\label{eq:newtonmotion}
\boldsymbol{a}(t) \equiv \frac{d\boldsymbol{v}(t)}{dt}=\frac{d^2\boldsymbol{r}(t)}{dt^2}=\frac{\boldsymbol{F}(t)}{m}\text{.}
\end{equation}
The Verlet algorithm was used to solve this ODE and in particular the velocity Verlet algorithm. There are two reasons for using this algorithm. Firstly it has a high order of accuracy $\mathcal{O}(h^3)$ and secondly it is very stable. The derivation for the velocity Verlet algorithm is as follows:\\
The Taylor expansions for the position:
\begin{equation}\label{eq:position}
\boldsymbol{r}(t+h) = \boldsymbol{r}(t) + h\frac{d\boldsymbol{r}(t)}{dt} + \frac{h^2}{2}\frac{d^2\boldsymbol{r}(t)}{dt^2}+\mathcal{O}(h^3)\text{.}
\end{equation}
Equation~(\ref{eq:position}) can be expressed in terms of velocity and force:
\begin{equation}\label{eq:position2}
\boldsymbol{r}(t+h) = \boldsymbol{r}(t) + h\boldsymbol{v}(t) + h^2\frac{\boldsymbol{F}(t)}{2m}+\mathcal{O}(h^3)\text{.}
\end{equation}
The Taylor expansion for velocity can be expressed as:
\begin{equation}\label{eq:velocity}
\boldsymbol{v}(t+h) = \boldsymbol{v}(t) + h\frac{d\boldsymbol{v}(t)}{dt} + \frac{h^2}{2}\frac{d^2\boldsymbol{v}(t)}{dt^2}+\mathcal{O}(h^3)\text{.}
\end{equation}
A second way to express the velocity in a Taylor expansion is:
\begin{equation}\label{eq:velocity2}
\boldsymbol{v}(t) = \boldsymbol{v}(t+h) - h\frac{d\boldsymbol{v}(t+h)}{dt} + \frac{h^2}{2}\frac{d^2\boldsymbol{v}(t+h)}{dt^2}+\mathcal{O}(h^3)\text{.}
\end{equation}
Subtracting Equation~(\ref{eq:velocity2}) from Equation~(\ref{eq:velocity}) can be expressed as:
\begin{equation}\label{eq:velocity3}
2\boldsymbol{v}(t+h)-2\boldsymbol{v}(t) =  h\frac{d\boldsymbol{v}(t+h)+d\boldsymbol{v}(t)}{dt} + \frac{h^2}{2}\frac{d^2\boldsymbol{v}(t)-d^2\boldsymbol{v}(t+h)}{dt^2}+\mathcal{O}(h^3)\text{.}
\end{equation}
If the Taylor expansion is taken for $\boldsymbol{v}(t+h)=\boldsymbol{v}(t)+\mathcal{O}(h)$ then the term $\frac{d^2\boldsymbol{v}(t)-d^2\boldsymbol{v}(t+h)}{dt^2}$ from Equation~(\ref{eq:velocity3}) reduces to $\mathcal{O}(h)$ and Equation~(\ref{eq:velocity3}) can be expressed as:
\begin{equation}\label{eq:velocity4}
\boldsymbol{v}(t+h)= \boldsymbol{v}(t) + \frac{h}{2}\frac{d\boldsymbol{v}(t+h)+d\boldsymbol{v}(t)}{dt} +\mathcal{O}(h^3)\text{.}
\end{equation}
So Equation(\ref{eq:velocity4}) can then be expressed in terms of force:
\begin{equation}\label{eq:velocityverlet}
\boldsymbol{v}(t+h)= \boldsymbol{v}(t) + \frac{h}{2m}(\boldsymbol{F}(t+h)+\boldsymbol{F}(t)) +\mathcal{O}(h^3)\text{.}
\end{equation}
\section{Boundary conditions}

\section{Initial conditions}

\end{document}



